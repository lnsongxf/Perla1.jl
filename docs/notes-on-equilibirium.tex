\documentclass[12pt]{article}
\usepackage{amsmath,amssymb,amsthm,verbatim,graphicx}
\usepackage{pifont}
\usepackage[hidelinks]{hyperref}
\usepackage{color}
\usepackage{xcolor}

\hypersetup{
	colorlinks,
	linkcolor={red!50!black},
	citecolor={red!50!black},
	urlcolor={blue!50!black}
}
\renewcommand{\P}{\mathbf{P}}
\newcommand{\Pn}{\mathbf{P_n}}
\newcommand{\Gn}{\mathbb{G}_n}
\newcommand{\E}{\mathbf{E}}
\newcommand{\R}{\mathbb{R}}
\newcommand{\Z}{\mathbb{Z}}
\newcommand{\N}{\mathbb{N}}
\newcommand{\C}{\mathbb{C}}
\newcommand{\cA}{\mathcal A}
\newcommand{\cB}{\mathcal B}
\newcommand{\cF}{\mathcal F}
\newcommand{\cG}{\mathcal G}
\newcommand{\cJ}{\mathcal J}
\newcommand{\cM}{\mathcal M}
\newcommand{\cN}{\mathcal N}
\newcommand{\cP}{\mathcal P}
\newcommand{\cU}{\mathcal U}
\newcommand{\cW}{\mathcal W}
\newcommand{\cX}{\mathcal X}
\newcommand{\cY}{\mathcal Y}
\newcommand{\cZ}{\mathcal Z}
\newcommand{\vone}{\mathbf 1}
\newcommand{\simple}{1}
\newcommand{\fisher}{\mathbb{I}}
\newcommand{\vA}{\mathbf A}
\newcommand{\vb}{\mathbf b}
\newcommand{\vdelta}{{\boldsymbol {\delta}}}
\newcommand{\ve}{\mathbf e}
\newcommand{\veta}{\boldsymbol \eta}
\newcommand{\vf}{\mathbf f}
\newcommand{\vG}{\mathbf G}
\newcommand{\vH}{\mathbf H}
\newcommand{\vs}{\mathbf s}
\newcommand{\vt}{\mathbf t}
\newcommand{\vv}{\mathbf v}
\newcommand{\vy}{\mathbf y}
\newcommand{\vyhat}{\hat{\mathbf y}}
\newcommand{\vD}{\mathbf D}
\newcommand{\vgamma}{{\boldsymbol {\gamma}}}
\newcommand{\vI}{\mathbf I}
\newcommand{\vM}{\mathbf M}
\newcommand{\vphi}{{\boldsymbol {\varphi}}}
\newcommand{\vQ}{\mathbf Q}
\newcommand{\vSigma}{\mathbf {\Sigma}}
\newcommand{\vT}{\mathbf T}
\newcommand{\vtheta}{{\boldsymbol {\theta}}}
\newcommand{\vTheta}{{\boldsymbol {\Theta}}}
\newcommand{\vU}{\mathbf U}
\newcommand{\vW}{\mathbf W}
\newcommand{\vX}{{\mathbf X}}
\newcommand{\vY}{{\mathbf Y}}
\newcommand{\vZ}{\mathbf Z}
\newcommand{\vbeta}{{\boldsymbol {\beta}}}
\newcommand{\vbetahat}{\widehat {\boldsymbol \beta}}
\newcommand{\vOmega}{\boldsymbol \Omega}
\newcommand{\sumiN}{\sum_{i = 1}^N}
\newcommand{\vmu}{\boldsymbol \mu}

\newcommand{\vp}{\mathbf p}
\newcommand{\vg}{\mathbf g}
\newcommand{\vx}{\mathbf x}
\newcommand{\vu}{\mathbf u}
\newcommand{\vw}{\mathbf w}
\newcommand{\vz}{\mathbf z}
\newcommand{\vzero}{\mathbf 0}

\newcommand{\convd}{\to_d}
\newcommand{\convp}{\to_P}

\newcommand{\eps}{\varepsilon}
\newcommand{\xbar}{\overline{x}}
\newcommand{\Xbar}{\overline{X}}
\newcommand{\ybar}{\overline{y}}
\newcommand{\Ybar}{\overline{Y}}
\newcommand{\sigmahat}{\widetilde{\sigma}}
\newcommand{\indep}{\perp}

\newcommand{\isumn}{\sum_{i=1}^n}
\newcommand{\iprodn}{\prod_{i=1}^n}
\DeclareMathOperator{\proj}{proj}
\DeclareMathOperator{\sgn}{sgn}
\DeclareMathOperator{\spt}{spt}
\DeclareMathOperator{\supp}{supp}
\DeclareMathOperator{\interior}{int}
\DeclareMathOperator{\Var}{Var}
\DeclareMathOperator{\Poi}{Poi}
\DeclareMathOperator{\Cov}{Cov}
\DeclareMathOperator{\Exp}{Exp}
\DeclareMathOperator{\vecop}{vec}
\DeclareMathOperator{\Geom}{Geom}

\newcommand{\cmark}{\ding{51}}%
\newcommand{\xmark}{\ding{55}}%
\newcommand{\done}{\rlap{$\square$}{\raisebox{2pt}{\large\hspace{1pt}\cmark}}%
	\hspace{-2.5pt}}
\newcommand{\wontfix}{\rlap{$\square$}{\large\hspace{1pt}\xmark}}
\newcommand{\progress}{\rlap{$\square$}{\large\hspace{1pt}\color{red}$\mathbf{\circ}$}}


\theoremstyle{definition}
\newtheorem{thm}{Proposition}[section]
\newtheorem{lem}{Lemma}[section]
\newtheorem{cor}{Corollary}[section]
\newtheorem{defn}{Definition}[section]
\newtheorem{remark}{Remarks}[section]
\newtheorem{example}{Examples}[section]
\newtheorem{algorithm}{Algorithm}[section]
\newtheorem{assumption}{Assumption}[section]

\newcommand{\thmautorefname}{Proposition}
\newcommand{\lemautorefname}{Lemma}
\newcommand{\defnautorefname}{Definition}
\newcommand{\remarkautorefname}{Remarks}
\newcommand{\exampleautorefname}{Example}
\newcommand{\algorithmautorefname}{Algorithm}
\newcommand{\assumptionautorefname}{Assumption}
\linespread{1.3}



%opening



\begin{document}
	

\noindent{\bf Notes on equilibrium (Chiyoung Ahn) \hfill } \\ 
\noindent\rule{\textwidth}{.1mm}
\section{Equilibrium with symmetric firms}
First, by Proposition 9, the total demand firm $i$ is facing at at age $a$ is 
\begin{align}
y_i(a, p_i, p_{-i}) &= \overline{\Gamma}^{1 - \kappa} \Omega q_i^{\kappa-1} p_i^{-1/\sigma-1} \E_a  \left[ \sum_{i' \in A} \left( \dfrac{q_i}{q_{i'}} p_{i'} \right)^{-1/\sigma}  \right]^{\sigma(\kappa-1) -1} \\
&= \overline{\Gamma}^{1 - \kappa} \Omega q_i^{\kappa-1} p_i^{-1/\sigma-1} \sum_{A | i } \left( \widehat f (a, A) \left[ \sum_{i' \in A} \left( \dfrac{q_i}{q_{i'}} p_{i'} \right)^{-1/\sigma}  \right]^{\sigma(\kappa-1) -1} \right) \label{eq:total-demand-general}
\end{align}

Assume that the awareness process is symmetric (as in paper), i.e., all firms have the same probability of being awared. By (3) in paper, we can then transform the expectation term, taken with respect to the awareness set, as the expectation with respect to the number of awared firms, $\widehat n$. 

Also, note that the probability of being included in the $n$ awared firms out of total $N$ firms is $n/N$.\footnote{In page 70, it is noted that 
	
	``First, note that in a symmetric equilibrium, finding the probability that a particular firm is in an awareness set of size $n$ with total $N$ firms is distributed Hypergeometric (i.e., an urn	problem without replacement).''
	
	This is actually true as long as the awareness process is symmetric (which is the case Perla considers throughout the paper) regardless of whether firms are symmetric or not.} Thus the mass of consumers who are aware of firm $i$ who have the size of $n$ is $(n/N )f_n(a)$. 

\textbf{Assume that firms are symmetric and the other firms choose the same price $\overline p$}. Then we can do noramlization on $q_i$ by setting $q_i = \overline{\Gamma}$ to simplify \autoref{eq:total-demand-general} to 
\begin{equation}\label{eq:total-demand-general-simple}
y_i(a, p_i, p_{-i}) = \Omega  p_i^{-1/\sigma-1} \sum_{A | i } \left( \widehat f (a, A) \left[ \sum_{i' \in A} \left( \dfrac{q_i}{q_{i'}} p_{i'} \right)^{-1/\sigma}  \right]^{\sigma(\kappa-1) -1} \right)
\end{equation}

Also, the summand given the awareness set $A$ with the size of $n$: 
$$
p^{-1/\sigma} + \sum_{i' \in A, i' \neq i} p_{i'}^{-1/\sigma}
$$
is identical with $p ^{-1/\sigma} + (n-1) \overline{p}^{-1/\sigma}$ by symmetry. The result follows with simple algebra.

\section{Equilibrium with asymmetric cohorts}
Let $K > 1$ be the number of cohorts, and assume that all cohorts have the same number of firms belong to each, $N$. Assume that there is a quality difference across cohort $q_{k}$, while keeping the awareness transition dynamics symmetric.  Since we have multiple cohorts, we introduce a $K-$length random vector $\widehat n_K$ whose $k$th element represents the number of firms awared in $k$th cohort, whose support is $\{0, 1,..., N\}$. Let $N_K$ be the support of $\widehat  n_K$, whose size is $(N+1)^K$.  For any $n_K \in N_K$, let $s(n_K)$ be defined as the elementwise sum of $n_K$ and let $n_K^k$ denote the $k$th element of $n_K$.

Note that, conditional on $n^k_K > 0$, the probability of a $k$-cohort firm being included in the $n^k_K$ awared firms out of total $N$ firms is $n^k_K/N$ (Porya urn). By (3) in paper, the expectation term in \autoref{eq:total-demand-general} can be written in terms of cross-cohort awareness counts as
\begin{equation}
\begin{aligned}
y_{ik}(a, p_i, p_{-i}) &=  \overline{\Gamma}^{1 - \kappa} \Omega q_k^{\kappa-1} p_i^{-1/\sigma-1} \dfrac{1}{N} \cdot \\ 
&\sum_{ n_K \in N_K } n^k_K f_{n_K}(a) \Bigg[ \sum_{k' =1 }^K n_K^{k'} \left( \dfrac{q_k}{q_{k'}} p_{k'} \right)^{-1/\sigma} +   
 p_i^{-1/\sigma}   - p^{-1/\sigma}_{k}  \Bigg]^{\sigma(\kappa-1) -1}  \\ 
 &=  \overline{\Gamma}^{1 - \kappa} \Omega q_k^{1/\sigma} p_i^{-1/\sigma-1} \dfrac{1}{N} \cdot \\ 
 &\sum_{ n_K \in N_K } n^k_K f_{n_K}(a) \Bigg[ \sum_{k' =1 }^K n_K^{k'} \left( \dfrac{p_{k'}}{q_{k'}}  \right)^{-1/\sigma} +   
 p_i^{-1/\sigma}   - p^{-1/\sigma}_{k}  \Bigg]^{\sigma(\kappa-1) -1}
\end{aligned}
\end{equation}
for firm $i$ on $k$th cohort. It is worth mentioning that the normalization as in \autoref{eq:total-demand-general-simple} is not possible if $q_k$ is different across cohorts  since $\overline{\Gamma}^{1-\kappa}  q_k^{\kappa-1} q_k^{1/\sigma}$ is not constant.


\end{document}
